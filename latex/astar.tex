% \documentclass[10pt]{scrartcl}
\documentclass[10pt,twocolumn]{scrartcl}

\usepackage[utf8]{inputenc}
\usepackage[T1]{fontenc}
\usepackage[ngerman]{babel}

\usepackage{amsmath}
\usepackage{amssymb}

\usepackage{graphicx}
\usepackage{tabularx}

\usepackage{algorithm}
\usepackage{algorithmic}

\setlength{\parindent}{0cm}
\setlength{\parskip}{3mm}
\setlength{\textheight}{23.8cm}
\setlength{\headheight}{1cm}
\setlength{\topmargin}{-10mm}

\setlength{\oddsidemargin}{0cm}
\setlength{\evensidemargin}{0cm}
\setlength{\textwidth}{16cm}
\setlength{\columnsep}{8mm}

\usepackage{multicol}
\usepackage{colortbl}
\usepackage{xcolor}
\definecolor{grau}{gray}{0.95}
\definecolor{dunkelgrau}{gray}{0.85}

\usepackage[normal]{caption}
\usepackage{lipsum}

\setlength{\parindent}{5mm}
\setlength{\parskip}{0mm}

\usepackage{float}
\restylefloat{figure}

\renewcommand{\topfraction}{0.75}
\renewcommand{\textfraction}{0.2}

%###########################################################
% die Sachen mit der Kopfzeile
\usepackage{lastpage}
\usepackage{fancyhdr}
\fancyhf{} % leere alle Felder
\fancyhead[R]{\footnotesize Sebastian Nocke, Luis Wientgens}
\fancyhead[L]{\footnotesize Der A* Suchalgorithmus} % Titel des Aufsatzes
\fancyfoot[C]{\footnotesize \thepage/\pageref{LastPage}}
% \fancyfoot[C]{\footnotesize \thepage}
\renewcommand{\headrulewidth}{0.4pt} % obere Trennlinie
\pagestyle{fancy}
%###########################################################

\newcommand{\ownsection}[1]{\begin{center}\LARGE\bf#1\end{center}}

\usepackage{hyperref}

\begin{document}

\twocolumn[
\ownsection{Der A* Suchalgorithmus}

\begin{center}
Sebastian Nocke, Luis Wientgens \\
Mannheim, April 2022
\end{center}
\vspace*{5mm}
]

% \begin{multicols}{2}
\begin{abstract}
Diese Ausarbeitung befasst sich mit dem als A* (sprich: a-star) bekannten Suchalgorithmus. Es werden dessen Einsatzgebiete erläutert und Eigenschaften diskutiert. Des Weiteren wird zum besseren Verständnis die Funktionsweise anhand eines Beispiels visualisiert.
\end{abstract}

\section*{Einleitung}

Bei A* handelt es sich um einen Wegfindungsalgorithmus.
Das Ziel von diesem ist es in einem gewichteten Graphen den kürzesten Weg von einem Startknoten zu einem Zielknoten zu finden.
Der Algorithmus benötigt zu den normalen Daten eine Graphs zusätzliche Informationen, daher benötigt er eine Heuristik, womit er in die Klasse der informierten Suchalgorithmen fällt.
Neben den Abständen der einzelnen Knoten muss die Position von jedem Knoten im Graphen bekannt sein.

Bei dem Weg, der durch den A* Algorithmus gefunden wird handelt es sich um eine optimale Lösung, es ist also garantiert, dass es sich um den kürzesten möglichen Weg handelt.

\section*{Anwendungsgebiete}

Der A* Algorithmus ist vielseitig einsetzbar.
Ein häufiges Anwendungsgebiet die die Wegplanung für autonome Roboter.
Diese verwendung liegt zudem nahe, da der Algorithmus zu diesem Zweck 1968 im Rahmen eines Robotikprojekts an der Stanford University entwickelt wurde \cite{b1}.

Neben der Robotik ist dieser Suchalgorithmus gut für Probleme geeignet, die sich in ein zweidimensionales Feld mit diskreten Punkten einteilen lassen.
Jeder dieser Punkte symbolisiert dabei eine Knoten des Graphen.
Zudem sind bei derartigen Problemen die Informationen über die Positionen der einzelnen Knoten gegeben.
Aufgrund dieser Gegebenheit findet der A* Algorithmus beziehungsweise angepasste Versionen von diesem häufig in Strategiespielen Verwendung.
Für diese Spiele, aber auch RTS-Spiele wie Warcraft III, in welchem A* verwendet wird, ist die Eigenschaft, dass der Algorithmus die optimale Lösung findet eine wichtige Eigenschaft.

A* wird jedoch nicht nur für die Traversierungsprobleme, für die der Algorithmus ursprünglich entwickelt wurde angewendet.
Eines der anderen Anwendungsgebiete ist das Parsen von stochastischen Grammatiken für Natural Language Processing.
Bei diesen Problemen handelt es sich um die Verarbeitung von menschlicher Sprache durch Computer.
In diesem Feld kann der A* Algorithmus verwendet werden, um Wörter und Sätze zu erkennen oder zu erzeugen.

\section*{Funktionsweise}

%Graph als Basis, Initialisierung der Listen, Iterativer Vorwärtschritt, Rekonstruktion des Pfades aus Vorgängerbeziehung, vielleicht als Forschungsfrage Manhattan Metrik und euklidische Metrik als Heuristik vergleichen? Einordnung in künstliche Intelligenz, da Fähigkeit zur Wegfindung Robotern gewisses Maß an Autonomie gibt. 

A* arbeitet auf einem gewichteten Graphen, wobei eine Lösung in einer Folge aus Knoten besteht, die den Pfad beschreiben. In unserer Implementierung wird der Graph als ein zweidimensionales Feld repräsentiert, dessen Elemente die Knoten darstellen. Um tatsächlicht ein Problem zu lösen, das über die triviale Situation eines leeren Feldes hinausgeht, dürfen einzelne Elemente des Feldes nicht traversierbar sein. Es gibt also zwei unterschiedliche Feldelemente die wir im Folgenden als \textit{Floor} für die traversierbaren und \textit{Wall} für die nicht traversierbaren bezeichnen.


Zunächst wird demnach zufälliges Feld mit Hindernissen als Umgebung generiert. Der Initialisierungschritt ist in Algorithmus \ref{alg:ainit} dargestellt. 

\begin{algorithm} 
\caption{A* Initialisierung} 
\label{alg:ainit} 
\begin{algorithmic}
	\REQUIRE width $w$, height $h$
	
	\STATE $field \leftarrow vector(w*h)$
	
	
	\FOR{$x \in 0...w$}	
	\FOR{$y \in 0...h$}

	\STATE $sample \leftarrow rand(0,1)$	
	\IF{$sample$}

	\STATE $vec[x,y] \leftarrow wall$	
	
	\ELSE
	
	\STATE $vec[x,y] \leftarrow floor$	
	
	\ENDIF	
	
	\ENDFOR
	\ENDFOR
\end{algorithmic}
\end{algorithm}

Auf dem generierten Feld wird schließlich die Wegfindung, der eigentliche A* ausgeführt. Zentrales Element dabei ist die sogenannte Evaluierungsfunktion. 

\begin{equation}
f(n) = g(n) + h(n)
\end{equation}

In Algorithmus \ref{alg:find} wird dieser Vorgang spezifiziert. Wichtig ist dabei, dass nur Feldelemente als Nachbarn beachtet werden, die auch traversierbar, daher kein Element vom Typ \textit{Wall} sind. Als Eingabe wird zusätzlich noch jeweils ein Element benötigt, das den Startknoten und den Zielknoten festlegt.

\begin{algorithm} 
\caption{A* Wegfindung} 
\label{alg:find} 
\begin{algorithmic}
	\REQUIRE point $start$, point $goal$, field $field$ 	
	
	\STATE new vector $open\_set$
	\STATE new map $predecessor\_map$ 
	
	\STATE add $start$ to $open\_set$
	
	\WHILE{$point \leftarrow open\_set.pop()$}
	\IF{$point = goal$}
	\STATE \textbf{return} $goal, predecessor\_map$	
	\ENDIF
	\STATE $neighbours \leftarrow get\_neighbours(point)$
	
	\FOR{$neighbour$ \textbf{in} $\neighbours$}
	
	\STATE $cost \leftarrow	g(neighbour)$
	
	\IF{\textbf{not} $neighbour$ \textbf{in} $predecessor\_map$}

	\STATE predecessor_map.insert(neighbour, point)
	
	\STATE add $neigbour$ to $open\_set$	
	
	\ENDIF
	
	\STATE sort $open\_set$ by lowest cost		
		
	\ENDFOR
	\ENDWHILE
\end{algorithmic}
\end{algorithm}

Damit steht nun eine Datenstruktur zur Verfügung, die einzelnen Elementen eine Vorgängerbeziehung zuordnet. Mittels des in Algorithmus \ref{alg:find} spezifizierten Vorgangs muss daher zunächst noch der Pfad rekonstruiert werden.

\begin{algorithm} 
\caption{A* Pfadrekonstruktion} 
\label{alg:rec} 
\begin{algorithmic}
	\REQUIRE point $goal$, map $predecessor\_map$
	
	\STATE new vector $steps$	
	\STATE $current \leftarrow goal$
	\STATE add $current$ to $steps$
	
	\WHILE{$next \leftarrow predecessor\_map.get(current)$}	
	
	\STATE $current \leftarrow next$
	\STATE add $current$ to $steps$

	\ENDWHILE
	\STATE reverse $steps$
	\STATE return $steps$
	
\end{algorithmic}
\end{algorithm}

Die finale Ausgabe ist schlussendlich ein Feld, dass die Elemente des optimalen Pfades in der Reihenfolge von Start zum Zielknoten enthält.

\section*{Eigenschaften}

Eine der wesentlichen Eigenschaften von A* ist die Optimalität, daher die Tatsache, dass der Algorithmus zu einem gegebenen Problem die optimale Lösung findet. Dies setzt voraus, dass eine geeignete Evaluierungsfunktion bzw. Heuristik verwendet wird \cite{b1}. 




\section*{Visualisierung}

Im Folgenden wird die Visualisierung des A* Algorithmus dargestellt.
Die beiden dargestellten Bilder sind durch das Programm, dass im Abschnitt Programmcode zu finden ist automatisch erstellt.
Bei der Visualisierung wird der gefundene, optimale Weg farblich gekennzeichnet.
Die weißen Felder im Bild sind diese, die begehbar sind und schwarze solche, über die nicht traversiert werden kann.
Zudem ist anzumerken, dass in der verwendeten Implementierung die Bewegung grundsätzlich in alle acht Richtungen möglich ist.

Der Startpunkt des visualisierten Weges ist rot gekennzeichnet, während der Zielknoten grün gefärbt ist.
Die Felder, die für den Weg abgegangen werden müssen stellen eine Farbübergang von rot zu grün da, je weiter sie vom Startpunkt entfernt werden.

Das erste Bild, dass das Ergebnis des Algorithmus darstellt ist auf ein 16 x 16 Feld angewendet.
Start- und Endkonten befinden sich in gegenüber liegenden Ecken des Feldes.
Diese Wahl der Punkte gilt ebenfalls für das zweite Beispiel, bei dem es sich jedoch um ein Feld mit einr Größe von 64 x 64 Feldern handelt.

\begin{figure}[!h]
	\center
	\includegraphics*[width=\columnwidth]{"images/small_path.png"}
\end{figure}

\begin{figure}[!h]
	\center
	\includegraphics*[width=\columnwidth]{"images/big_path.png"}
\end{figure}

\section*{Fazit}

In vielen Einsatzgebieten gehört der A* mittlerweile zum Standardkanon der Algorithmik. Insbesondere in der Robotik dürfte niemand mehr daran vorbeikommen, sich zumindest grundlegend mit dessen Funktionsweise zu beschäftigen. Auch dient er als Basis für eine ganze Reihe an optimierten und spezialisierten Algorithmen, die in verschiedenen Bereichen Verwendung finden. Damit hat A* trotz seines Alters bis heute eine hohe praktische Relevanz.

\section*{Programmcode}

Die hier beschriebene Implementierung ist auf GitHub unter dem Link \url{https://github.com/CIsMyPassion/a\_star} zu finden.

% \end{multicols}


\begin{thebibliography}{99}
\bibitem{b1} P. E. Hart, N. J. Nilsson und B. Raphael, ``A Formal Basis for the Heuristic Determination of Minimum Cost Paths'' IEEE Transactions of Systems  Science and Cybernetics, Vol. 4, No. 2, Juli 1968.
\bibitem{b2} W. Zeng und R.L. Church, ``Finding shortest paths on real road networks: the case for A*'' International Journal of Geographical Information Science, Vol. 23, No. 4, April 2009.

\end{thebibliography}

\end{document}
