% \documentclass[10pt]{scrartcl}
\documentclass[10pt,twocolumn]{scrartcl}

\usepackage[utf8]{inputenc}
\usepackage[T1]{fontenc}
\usepackage[ngerman]{babel}

\usepackage{amsmath}
\usepackage{amssymb}

\usepackage{graphicx}
\usepackage{tabularx}

\setlength{\parindent}{0cm}
\setlength{\parskip}{3mm}
\setlength{\textheight}{23.8cm}
\setlength{\headheight}{1cm}
\setlength{\topmargin}{-10mm}

\setlength{\oddsidemargin}{0cm}
\setlength{\evensidemargin}{0cm}
\setlength{\textwidth}{16cm}
\setlength{\columnsep}{8mm}

\usepackage{multicol}
\usepackage{colortbl}
\usepackage{xcolor}
\definecolor{grau}{gray}{0.95}
\definecolor{dunkelgrau}{gray}{0.85}

\usepackage[normal]{caption}
\usepackage{lipsum}

\setlength{\parindent}{5mm}
\setlength{\parskip}{0mm}

\usepackage{float}
\restylefloat{figure}

\renewcommand{\topfraction}{0.75}
\renewcommand{\textfraction}{0.2}

%###########################################################
% die Sachen mit der Kopfzeile
\usepackage{lastpage}
\usepackage{fancyhdr}
\fancyhf{} % leere alle Felder
\fancyhead[R]{\footnotesize Sebastian Nocke, Luis Wientgens}
\fancyhead[L]{\footnotesize Der A* Suchalgorithmus} % Titel des Aufsatzes
\fancyfoot[C]{\footnotesize \thepage/\pageref{LastPage}}
% \fancyfoot[C]{\footnotesize \thepage}
\renewcommand{\headrulewidth}{0.4pt} % obere Trennlinie
\pagestyle{fancy}
%###########################################################

\newcommand{\ownsection}[1]{\begin{center}\LARGE\bf#1\end{center}}

\begin{document}

\twocolumn[
\ownsection{Der A* Suchalgorithmus}

\begin{center}
Sebastian Nocke, Luis Wientgens \\
Mannheim, April 2022
\end{center}
\vspace*{5mm}
]

% \begin{multicols}{2}
\section*{Abstract}

Such Such 

\section*{Einleitung}


\section*{Anwendungsgebiete}

\section*{Pseudocode}

\section*{Eigenschaften}


\section*{Visualisierung}

\section*{Fazit}
% \end{multicols}


\begin{thebibliography}{99}
\bibitem{Goos1947}F.Goos und H.Hänchen: {\it Ein neuer fundamentaler Versuch zur Totalreflektion}, 1947; Annalen der Physik 436, S. 333-346
\bibitem{Einstein1905}A.Einstein {\it Zur Elektrodynamik bewegter Körper}, 1905; Annalen der Physik und Chemie 17, S. 891-921
\bibitem{Gerhards2008}H.Gerhards: {\it Ground Penetrating Radar as a Quantative Tool with Applications in Soil Hydrology}, Heidelberg 2008; Dissertaton,
\bibitem{Wittgenstein1922}L.Wittgenstein: {\it Tractatus Logico-Philo\-so\-phi\-cus}, London 1922; Kegan Paul, Trench, Trubner \& Co., Ltd.
\end{thebibliography}

\end{document}