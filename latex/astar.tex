% \documentclass[10pt]{scrartcl}
\documentclass[10pt,twocolumn]{scrartcl}

\usepackage[utf8]{inputenc}
\usepackage[T1]{fontenc}
\usepackage[ngerman]{babel}

\usepackage{amsmath}
\usepackage{amssymb}

\usepackage{graphicx}
\usepackage{tabularx}

\usepackage{algorithm}
\usepackage{algorithmic}

\setlength{\parindent}{0cm}
\setlength{\parskip}{3mm}
\setlength{\textheight}{23.8cm}
\setlength{\headheight}{1cm}
\setlength{\topmargin}{-10mm}

\setlength{\oddsidemargin}{0cm}
\setlength{\evensidemargin}{0cm}
\setlength{\textwidth}{16cm}
\setlength{\columnsep}{8mm}

\usepackage{multicol}
\usepackage{colortbl}
\usepackage{xcolor}
\definecolor{grau}{gray}{0.95}
\definecolor{dunkelgrau}{gray}{0.85}

\usepackage[normal]{caption}
\usepackage{lipsum}

\setlength{\parindent}{5mm}
\setlength{\parskip}{0mm}

\usepackage{float}
\restylefloat{figure}

\renewcommand{\topfraction}{0.75}
\renewcommand{\textfraction}{0.2}

%###########################################################
% die Sachen mit der Kopfzeile
\usepackage{lastpage}
\usepackage{fancyhdr}
\fancyhf{} % leere alle Felder
\fancyhead[R]{\footnotesize Sebastian Nocke, Luis Wientgens}
\fancyhead[L]{\footnotesize Der A* Suchalgorithmus} % Titel des Aufsatzes
\fancyfoot[C]{\footnotesize \thepage/\pageref{LastPage}}
% \fancyfoot[C]{\footnotesize \thepage}
\renewcommand{\headrulewidth}{0.4pt} % obere Trennlinie
\pagestyle{fancy}
%###########################################################

\newcommand{\ownsection}[1]{\begin{center}\LARGE\bf#1\end{center}}

\begin{document}

\twocolumn[
\ownsection{Der A* Suchalgorithmus}

\begin{center}
Sebastian Nocke, Luis Wientgens \\
Mannheim, April 2022
\end{center}
\vspace*{5mm}
]

% \begin{multicols}{2}
\begin{abstract}
Diese Ausarbeitung befasst sich mit dem als A* (sprich: a-star) bekannten Suchalgorithmus. Es werden dessen Einsatzgebiete erläutert und Eigenschaften diskutiert. Des Weiteren wird zum besseren Verständnis die Funktionsweise anhand eines Beispiels visualisiert.
\end{abstract}

\section*{Einleitung}
Bei A* handelt es sich um einen Wegfindungsalgorithmus.
Das Ziel von diesem ist es in einem informiertem Graphen den kürzesten Weg von einem Startknoten zu einem Zielknoten zu finden.
Der Algorithmus benötigt zu den normalen Daten eine Graphs zusätzliche Informationen, daher benötigt er einen informierten Graph.
Neben den Abständen der einzelnen Knoten muss die Position von jedem Knoten im Graphen bekannt sein.

Bei dem Weg, der durch den A* Algorithmus gefunden wird handelt es sich um eine optimale Lösung, es ist also garantiert, dass es sich um den kürzesten möglichen Weg handelt.

\section*{Anwendungsgebiete}
Der A* Algorithmus ist vielseitig einsetzbar.
Ein häufiges Anwendungsgebiet die die Wegplanung für autonome Roboter.
Diese verwendung liegt zudem nahe, da der Algorithmus zu diesem Zweck entwickelt wurde.

Neben der Robotik ist dieser Suchalgorithmus gut für Probleme geeignet, die sich in ein zweidimensionales Feld mit diskreten Punkten einteilen lassen.
Jeder dieser Punkte symbolisiert dabei eine Knoten des Graphen.
Zudem sind bei derartigen Problemen die Informationen über die Positionen der einzelnen Knoten gegeben.
Aufgrund dieser Gegebenheit ist der A* Algorithmus beziehungsweise angepasste Versionen von diesem häufig in Strategiespielen verwenden.
Für diese Spiele, aber auch Spiele wie Warcraft III, in welchem A* verwendet wird ist die Eigenschaft, dass der Algorithmus die optimale Lösung findet eine wichtige Eigenschaft.

A* wird jedoch nicht nur für die Traversierungsprobleme, für die der Algorithmus ursprünglich entwickelt wurde.
Einses dieser Anwendungsgebiete ist das parsen von stochastischen Grammatiken für Natural Language Processing.
Bei diesen Problemen handelt es sich um die Verarbeitung von menschlicher Sprache durch Computer.
In diesem Feld kann der A* Algorithmus verwendet werden, um Wörter und Sätze zu erkennen oder zu erzeugen.

\section*{Funktionsweise}

%Graph als Basis, Initialisierung der Listen, Iterativer Vorwärtschritt, Rekonstruktion des Pfades aus Vorgängerbeziehung, vielleicht als Forschungsfrage Manhattan Metrik und euklidische Metrik als Heuristik vergleichen? Einordnung in künstliche Intelligenz, da Fähigkeit zur Wegfindung Robotern gewisses Maß an Autonomie gibt. 


Zunächst wird ein zufälliges Feld mit Hindernissen als Umgebung generiert. Der Initialisierungschritt ist in Algorithmus \ref{alg:ainit} dargestellt. 

\begin{algorithm} 
\caption{A* Initialisierung} 
\label{alg:ainit} 
\begin{algorithmic}
	\REQUIRE width $w$, height $h$
	
	\STATE $field \leftarrow vector(w*h)$
	
	
	\FOR{$x \in 0...w$}	
	\FOR{$y \in 0...h$}

	\STATE $sample \leftarrow rand(0,1)$	
	\IF{$sample$}

	\STATE $vec[x,y] \leftarrow wall$	
	
	\ELSE
	
	\STATE $vec[x,y] \leftarrow floor$	
	
	\ENDIF	
	
	\ENDFOR
	\ENDFOR
\end{algorithmic}
\end{algorithm}

Auf dem generierten Feld wird die Suchfunktion des A* ausgeführt. In Algorithmus \ref{alg:find} wird der Suchschritt gezeigt.

\begin{algorithm} 
\caption{A* Wegfindung} 
\label{alg:find} 
\begin{algorithmic}
	\REQUIRE point $start$, point $goal$, field $field$ 	
	
	\STATE new vector $open\_set$
	\STATE new map $predecessor\_map$ 
	
	\STATE add $start$ to $open\_set$
	
	\WHILE{$point \leftarrow open\_set.pop()$}
	\IF{$point = goal$}
	\STATE \textbf{return} $goal, predecessor\_map$	
	\ENDIF
	\STATE $neighbours \leftarrow get\_neighbours(point)$
	
	\FOR{$neighbour$ \textbf{in} $\neighbours$}
	
	\STATE $cost \leftarrow	g(neighbour)$
	
	\IF{\textbf{not} $neighbour$ \textbf{in} $predecessor\_map$}

	\STATE predecessor_map.insert(neighbour, point)
	
	\STATE add $neigbour$ to $open\_set$	
	
	\ENDIF
	
	\STATE sort $open\_set$ by lowest cost		
		
	\ENDFOR
	\ENDWHILE
\end{algorithmic}
\end{algorithm}

Mittels des in Algorithmus \ref{alg:find} spezifizierten Vorgangs wird der Pfad rekonstruiert.

\begin{algorithm} 
\caption{A* Pfadrekonstruktion} 
\label{alg:rec} 
\begin{algorithmic}
	\REQUIRE point $goal$, map $predecessor\_map$
	
	\STATE new vector $steps$	
	\STATE $current \leftarrow goal$
	\STATE add $current$ to $steps$
	
	\WHILE{$next \leftarrow predecessor\_map.get(current)$}	
	
	\STATE $current \leftarrow next$
	\STATE add $current$ to $steps$

	\ENDWHILE
	\STATE reverse $steps$
	\STATE return $steps$
	
\end{algorithmic}
\end{algorithm}

\section*{Eigenschaften}






\section*{Visualisierung}

\section*{Fazit}

In vielen Einsatzgebieten gehört der A* mittlerweile zum Standardkanon der Algorithmik. Insbesondere in der Robotik dürfte niemand mehr daran vorbeikommen, sich zumindest grundlegend mit dessen Funktionsweise zu beschäftigen.

% \end{multicols}


\begin{thebibliography}{99}
\bibitem{b1} P. E. Hart, N. J. Nilsson und B. Raphael, ``A Formal Basis for the Heuristic Determination of Minimum Cost Paths'' IEEE Transactions of Systems  Science and Cybernetics, Vol. 4, No. 2, Juli 1968.
\bibitem{b2} W. Zeng und R.L. Church, ``Finding shortest paths on real road networks: the case for A*'' International Journal of Geographical Information Science, Vol. 23, No. 4, April 2009.

\end{thebibliography}

\end{document}
